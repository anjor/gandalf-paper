\section{Verification and Validation}
\label{sec:verification}

We verify GANDALF's implementation through a hierarchy of benchmarks progressing from linear wave propagation to nonlinear dynamics. This section presents the Alfvén wave dispersion test, which validates both the spectral discretization and the GANDALF time integration scheme against analytical predictions.

\subsection{Linear Physics: Alfvén Wave Dispersion}
\label{sec:alfven_wave}

\subsubsection{Analytical Prediction}

For a single Fourier mode with wavenumber $\kvec = (0, 0, \kpar)$ aligned with the guide field, Eqs.~\eqref{eq:elsasser} reduce to the linear Alfvén wave dispersion relation \eqref{eq:alfven_dispersion}:
\begin{equation}
\omega = \kpar \vA
\end{equation}
This relation holds exactly in the KRMHD regime where $\kperp \rhoi \ll 1$ and kinetic modifications from finite Larmor radius effects remain negligible. The wave propagates without distortion or damping in the collisionless limit ($\eta = \nu = \coll = 0$), providing a clean test of numerical accuracy.

The Alfvén wave represents the fundamental mode of magnetized plasma dynamics. In the Elsasser formulation, a pure parallel-propagating wave ($\kperp = 0$) couples the forward and backward Els asser fields $\Zp$ and $\Zm$ through the parallel gradient term $\dpar\Zm$ in \Cref{eq:elsasser}, producing oscillations at frequency $\omega = |\kpar|\vA$. The perpendicular structure remains frozen, making this an ideal benchmark for time integration accuracy independent of perpendicular spectral operations.

\subsubsection{Numerical Setup}

We initialize a single Fourier mode at $\kvec = (0, 0, 1)$ with amplitude $A = 0.01$ (linear regime) on a periodic domain of size $L_x = L_y = L_z = 2\pi$. The Elsasser fields are initialized with correct phase relationship for a circularly polarized Alfvén wave: $\Zp(\kvec) = A \cdot i$ and $\Zm(\kvec) = A$, ensuring the wave contains both forward and backward propagating components necessary for coupled oscillations.

The parallel wavenumber $\kpar = 2\pi/L_z = 1$ with $\vA = 1$ yields analytical frequency $\omega_{\mathrm{analytical}} = 1$. We integrate for 5--10 Alfvén periods ($t = 5$--$10 \times 2\pi/\omega \approx 31$--$63$) to accumulate sufficient phase evolution for accurate frequency measurement while remaining well within the numerical stability regime.

Physical parameters are $\betai = 1$, $\vth = 1$, and $M = 20$ Hermite moments with all dissipation coefficients set to zero ($\eta = \nu = \coll = 0$). The simulation tracks the complex amplitude of the Fourier mode $\hat{\Zp}(\kpar, t)$ at each timestep, measuring frequency via phase evolution:
\begin{equation}
\hat{\Zp}(\kpar, t) = A(t) e^{i\phi(t)}, \quad \omega_{\mathrm{measured}} = \frac{d\phi}{dt}
\end{equation}
where the phase $\phi(t)$ is unwrapped to handle $2\pi$ discontinuities and fitted linearly in time to extract the oscillation frequency.

\subsubsection{Convergence Results}

\paragraph{Spatial Convergence}
We measure dispersion relation accuracy as a function of grid resolution $N$ (for $N^3$ grids) while holding the timestep fixed at $\Delta t = 0.01$ (well-resolved for all tested resolutions). \Cref{fig:alfven_convergence}(a) shows exponential convergence of the relative frequency error $|\omega_{\mathrm{measured}} - \omega_{\mathrm{analytical}}| / \omega_{\mathrm{analytical}}$ with increasing resolution.

Fitting the convergence data to $E = A e^{-\alpha N}$ yields exponential rate $\alpha \approx 0.XXX$, characteristic of spectral methods \citep{Boyd2001}. At resolution $N = 256^3$, the relative error reaches $\sim 10^{-X}$, demonstrating that GANDALF achieves spectral accuracy for smooth problems. The exponential convergence confirms correct implementation of the Fourier spectral operators and absence of aliasing errors from the $2/3$-rule dealiasing.

For comparison, finite-difference methods achieve at best algebraic convergence $E \sim N^{-p}$ with $p \leq 4$ for standard schemes \citep{LeVeque2007}, requiring orders of magnitude more grid points to reach equivalent accuracy. The spectral approach proves essential for resolving the broad range of scales in plasma turbulence cascades.

\paragraph{Temporal Convergence}
Fixing spatial resolution at $N = 64^3$ (sufficient to resolve the $\kpar = 1$ mode), we vary the timestep $\Delta t$ from $T/10$ to $T/160$ where $T = 2\pi/\omega$ is the wave period. \Cref{fig:alfven_convergence}(b) demonstrates second-order temporal convergence $E \sim (\Delta t)^2$, confirming the expected accuracy of the GANDALF integrating factor combined with RK2 midpoint rule for nonlinear terms.

Fitting the measured errors to $E = C\, (\Delta t)^p$ yields power law exponent $p \approx 2.XX$, in excellent agreement with the theoretical prediction $p = 2$ for the RK2 scheme. The GANDALF integrating factor treats linear Alfvén wave propagation exactly (removing the Alfvén CFL constraint entirely), while the midpoint rule provides second-order accuracy for nonlinear interactions.

Crucially, the temporal errors remain modest even at the largest tested timestep $\Delta t = 2\pi/10 \approx 0.628$, which corresponds to resolving the wave period with only 10 points. Standard explicit schemes would require $\Delta t \lesssim \Delta z/\vA \sim 2\pi/(64 \times 1) \approx 0.098$ for stability, making GANDALF approximately $6\times$ faster for this problem. The stability advantage increases further for higher resolutions where the CFL timestep scales as $\Delta t \sim N^{-1}$.

\paragraph{Dispersion Relation Validation}
\Cref{fig:alfven_dispersion} plots measured frequency against analytical prediction for all benchmark runs (spatial and temporal convergence studies combined). All data points lie on the $\omega_{\mathrm{measured}} = \omega_{\mathrm{analytical}}$ line within error bars, with maximum relative error $\sim 10^{-X}$ and mean error $\sim 10^{-X}$ across all configurations.

The tight clustering of points around the identity line confirms that GANDALF correctly implements the KRMHD equations with no systematic bias in wave propagation speed. Both the real and imaginary parts of the Elsasser field evolution match analytical solutions, verifying phase and amplitude accuracy simultaneously.

\subsubsection{Discussion}

The Alfvén wave benchmark demonstrates three key properties of GANDALF:

\begin{enumerate}
\item \textbf{Spectral spatial accuracy}: Exponential convergence with resolution confirms correct implementation of Fourier spectral derivatives and dealiasing. For smooth solutions, spectral methods achieve machine precision with modest grid sizes, making them ideal for resolving turbulent cascades extending over multiple decades in wavenumber.

\item \textbf{Integrating factor advantage}: Second-order temporal convergence without Alfvén CFL constraint enables larger timesteps than explicit schemes, improving computational efficiency for problems where $\vA \gg v_{\mathrm{rms}}$. The $\sim 6\times$ speedup observed here compounds with the parallelizability of spectral transforms on GPUs/TPUs.

\item \textbf{Linear wave fidelity}: Accurate reproduction of $\omega = \kpar\vA$ validates the parallel derivative operator, Elsasser field coupling, and time integration of wave propagation. This provides confidence for simulations of Alfvénic turbulence where linear wave physics competes with nonlinear interactions.
\end{enumerate}

Future benchmarks will extend validation to nonlinear regimes (Orszag-Tang vortex), compressive physics (slow mode damping), and kinetic effects (Landau damping with finite $\kperp\rhoi$). For applications to weakly collisional plasmas like the solar wind, the linear wave benchmark establishes baseline accuracy before introducing the complexities of phase mixing, resonant dissipation, and intermittent structures \citep{Schekochihin2009,Meyrand2019}.

\begin{figure}[t]
\centering
\includegraphics[width=\textwidth]{figures/alfven_frequency_convergence.pdf}
\caption{Alfvén wave dispersion benchmark convergence studies. (a) Spatial convergence: relative frequency error versus grid resolution $N$ (for $N^3$ grids) with fixed timestep $\Delta t = 0.01$. The exponential decay $E \sim e^{-\alpha N}$ (dashed line, $\alpha \approx 0.XXX$) demonstrates spectral accuracy. (b) Temporal convergence: relative error versus timestep $\Delta t$ with fixed resolution $N = 64^3$. The power law $E \sim (\Delta t)^p$ (dashed line, $p \approx 2.XX$) confirms second-order accuracy of the GANDALF integrating factor with RK2. Gray dotted line shows reference $O(\Delta t^2)$ scaling.}
\label{fig:alfven_convergence}
\end{figure}

\begin{figure}[t]
\centering
\includegraphics[width=0.65\textwidth]{figures/alfven_dispersion_validation.pdf}
\caption{Measured Alfvén wave frequency versus analytical prediction $\omega = \kpar\vA$ for all benchmark configurations (circles: spatial convergence at varying $N$; squares: temporal convergence at varying $\Delta t$). All points lie on the identity line (dashed) within numerical error, with maximum relative deviation $\sim 10^{-X}$. This confirms correct implementation of the KRMHD dispersion relation across a wide range of numerical parameters.}
\label{fig:alfven_dispersion}
\end{figure}
