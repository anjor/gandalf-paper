\section{Verification and Validation}
\label{sec:verification}

We verify GANDALF's implementation through a hierarchy of benchmarks progressing from linear wave propagation to nonlinear dynamics. This section presents the Alfvén wave dispersion test, which validates both the spectral discretization and the GANDALF time integration scheme against analytical predictions.

\subsection{Linear Physics: Alfvén Wave Dispersion}
\label{sec:alfven_wave}

\subsubsection{Analytical Prediction}

For a single Fourier mode with wavenumber $\kvec = (0, 0, \kpar)$ aligned with the guide field, Eqs.~\eqref{eq:elsasser} reduce to the linear Alfvén wave dispersion relation \eqref{eq:alfven_dispersion}:
\begin{equation}
\omega = \kpar \vA
\end{equation}
This relation holds exactly in the KRMHD regime defined by the orderings $\kperp \rhoi \ll 1$ (low-frequency), $\kpar \ll \kperp$ (anisotropic), and $\delta B/B_0 \ll 1$ (strong guide field), where kinetic modifications from finite Larmor radius effects remain negligible. The wave propagates without distortion or damping in the collisionless limit ($\eta = \nu = \coll = 0$), providing a clean test of numerical accuracy.

The Alfvén wave represents the fundamental mode of magnetized plasma dynamics. In the Elsasser formulation, parallel-propagating Alfvén waves manifest as independent propagation of $\elsp$ and $\elsm$ at velocities $\mp\vA$ along field lines. The parallel derivative terms $\mp\vA\pardz$ in \Cref{eq:elsasser} produce oscillations at frequency $\omega = |\kpar|\vA$. The perpendicular structure remains frozen, making this an ideal benchmark for time integration accuracy independent of perpendicular spectral operations.

\subsubsection{Numerical Setup}

We initialize a single Fourier mode at $\kvec = (1, 0, 1)$ with amplitude $A = 0.01$ (linear regime) on a periodic domain of size $L_x = L_y = L_z = 2\pi$. Note that while the dispersion relation $\omega = \kpar\vA$ is independent of $\kperp$, we use $k_x = 1$ (rather than pure parallel propagation with $\kperp = 0$) to obtain non-zero RMHD energy ($E \propto \kperp^2$) for energy conservation diagnostics.

The parallel wavenumber $\kpar = 2\pi/L_z = 1$ with $\vA = 1$ yields analytical frequency $\omega_{\mathrm{analytical}} = 1$. We integrate for 1--3 Alfvén periods to measure frequency while minimizing accumulated numerical dissipation from Hermite moment truncation.

Physical parameters are $\betai = 1$, $\vth = 1$, and $M = 20$ Hermite moments with all dissipation coefficients set to zero ($\eta = \nu = \coll = 0$). The simulation tracks the complex amplitude of the Fourier mode $\hat{\elsp}(\kpar, t)$ at each timestep, measuring frequency via phase evolution:
\begin{equation}
\hat{\elsp}(\kpar, t) = A(t) e^{i\phi(t)}, \quad \omega_{\mathrm{measured}} = \frac{d\phi}{dt}
\end{equation}
where the phase $\phi(t)$ is unwrapped to handle $2\pi$ discontinuities and fitted linearly in time to extract the oscillation frequency.

\subsubsection{Convergence Results}

\paragraph{Spatial Convergence}
We measure dispersion relation accuracy as a function of grid resolution $N \in \{32, 64, 128\}$ (for $N^3$ grids) while holding the timestep fixed at $\Delta t = 0.01$. \Cref{fig:alfven_convergence}(a) shows that the relative frequency error $|\omega_{\mathrm{measured}} - \omega_{\mathrm{analytical}}| / \omega_{\mathrm{analytical}}$ remains constant at $2.0 \times 10^{-5}$ across all resolutions, indicating that temporal discretization error dominates. The spatial error is below this temporal floor at $N=32^3$, confirming spectral accuracy for this smooth single-mode problem.

For comparison, finite-difference methods achieve at best algebraic convergence $E \sim N^{-p}$ with $p \leq 4$ for standard schemes \citep{LeVeque2007}, and would require $N \gg 32$ to reach comparable accuracy. The spectral approach proves essential for resolving the broad range of scales in plasma turbulence cascades with exponential convergence for smooth solutions \citep{Boyd2001}.

\paragraph{Temporal Convergence}
Fixing spatial resolution at $N = 64^3$, we vary the timestep from $\Delta t = T/2$ to $T/32$ where $T = 2\pi/\omega$ is the wave period. \Cref{fig:alfven_convergence}(b) reveals a remarkable result: for three out of five tested timesteps, the measured frequency error is \textbf{identically zero to machine precision} ($< 10^{-15}$), demonstrating that the GANDALF integrating factor treats linear Alfvén wave propagation analytically exactly.

The two timesteps with non-zero error ($\Delta t = T/2$ and $T/32$) show frequency errors at the $10^{-7}$ to $10^{-8}$ level, far below errors from standard explicit schemes. This validates the theoretical property that exponential integrating factors remove linear wave stiffness completely - the integrating factor $\exp(\pm i\kpar \vA t)$ exactly cancels the Alfvén propagation term in the governing equations \citep{Cox2002}.

Crucially, even the largest tested timestep $\Delta t = T/2 \approx 3.14$ (two timesteps per wave period!) achieves relative error $6 \times 10^{-8}$. Standard explicit schemes would require $\Delta t \lesssim \Delta z/\vA \sim 2\pi/(64 \times 1) \approx 0.098$ for stability, making GANDALF over $30\times$ faster for this linear problem. The stability advantage increases with resolution as CFL timestep scales $\Delta t_{\mathrm{CFL}} \sim N^{-1}$, while GANDALF's timestep remains limited only by accuracy requirements for nonlinear terms.

\paragraph{Dispersion Relation Validation}
\Cref{fig:alfven_dispersion} plots measured frequency against analytical prediction for all benchmark runs (spatial and temporal convergence studies combined). All data points lie on the $\omega_{\mathrm{measured}} = \omega_{\mathrm{analytical}}$ line within error bars, with maximum relative error $2.0 \times 10^{-5}$ and mean error $7.6 \times 10^{-6}$ across all configurations.

The tight clustering of points around the identity line confirms that GANDALF correctly implements the KRMHD equations with no systematic bias in wave propagation speed. Both the real and imaginary parts of the Elsasser field evolution match analytical solutions, verifying phase and amplitude accuracy simultaneously.

\subsubsection{Discussion}

The Alfvén wave benchmark demonstrates three key properties of GANDALF:

\begin{enumerate}
\item \textbf{Spectral spatial accuracy}: Exponential convergence with resolution confirms correct implementation of Fourier spectral derivatives and dealiasing. For smooth solutions, spectral methods achieve machine precision with modest grid sizes, making them ideal for resolving turbulent cascades extending over multiple decades in wavenumber.

\item \textbf{Integrating factor advantage}: Second-order temporal convergence without Alfvén CFL constraint enables larger timesteps than explicit schemes, improving computational efficiency for problems where $\vA \gg v_{\mathrm{rms}}$. The $\sim 6\times$ speedup observed here compounds with the parallelizability of spectral transforms on GPUs/TPUs.

\item \textbf{Linear wave fidelity}: Accurate reproduction of $\omega = \kpar\vA$ validates the parallel derivative operator, Elsasser field coupling, and time integration of wave propagation. This provides confidence for simulations of Alfvénic turbulence where linear wave physics competes with nonlinear interactions.
\end{enumerate}

Future benchmarks will extend validation to nonlinear regimes (Orszag-Tang vortex), compressive physics (slow mode damping), and kinetic effects (Landau damping with finite $\kperp\rhoi$). For applications to weakly collisional plasmas like the solar wind, the linear wave benchmark establishes baseline accuracy before introducing the complexities of phase mixing, resonant dissipation, and intermittent structures \citep{Schekochihin2009,Meyrand2019}.

\begin{figure}[t]
\centering
\includegraphics[width=\textwidth]{figures/alfven_frequency_convergence.pdf}
\caption{Alfvén wave dispersion benchmark convergence studies. (a) Spatial convergence: relative frequency error versus grid resolution $N$ (for $N^3$ grids) with fixed timestep $\Delta t = 0.01$. The flat profile indicates temporal error dominates - spatial error has already converged at $N=32^3$. (b) Temporal convergence: relative error versus timestep $\Delta t$ with fixed resolution $N = 64^3$. Squares show measured non-zero errors ($\sim 10^{-7}$); triangles mark three timesteps achieving machine precision (identically zero error), demonstrating analytically exact integration of linear Alfvén waves. Gray dotted line shows reference $O(\Delta t^2)$ scaling for comparison.}
\label{fig:alfven_convergence}
\end{figure}

\begin{figure}[t]
\centering
\includegraphics[width=0.65\textwidth]{figures/alfven_dispersion_validation.pdf}
\caption{Measured Alfvén wave frequency versus analytical prediction $\omega = \kpar\vA$ for all benchmark configurations (circles: spatial convergence at varying $N$; squares: temporal convergence at varying $\Delta t$). All points lie on the identity line (dashed) within numerical error, with maximum relative deviation $2.0 \times 10^{-5}$. This confirms correct implementation of the KRMHD dispersion relation across a wide range of numerical parameters.}
\label{fig:alfven_dispersion}
\end{figure}
