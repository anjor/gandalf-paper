\section{Introduction}

\Alfvenic\ turbulence governs energy transport in diverse magnetized plasma environments, from the solar wind and corona to tokamak fusion devices. Recent Parker Solar Probe observations reveal large-amplitude \Alfven\ waves heating and accelerating the nascent solar wind \citep{Rivera2024}, while measurements in the heliosphere demonstrate complex energy transfer through imbalanced \Alfvenic\ turbulence \citep{Yang2023}. Understanding these phenomena requires bridging magnetohydrodynamic (MHD) cascade physics with kinetic dissipation mechanisms---a regime naturally captured by Kinetic Reduced MHD (KRMHD). KRMHD describes anisotropic turbulence in strongly magnetized plasmas where perpendicular wavenumbers dominate ($\kpar \ll \kperp$), retaining essential kinetic physics through Landau damping and phase mixing while avoiding the computational expense of full gyrokinetic treatments \citep{Schekochihin2009,GoldreichSridhar1995}. This intermediate framework enables quantitative studies of turbulent cascades, energy dissipation, and the interplay between \Alfvenic\ and compressive fluctuations that characterize weakly collisional plasma turbulence across astrophysical and laboratory settings.

The KRMHD equations emerge from systematic expansion of the gyrokinetic system in the limit of small perpendicular ion Larmor radius ($\kperp \rhoi \ll 1$) and strong guide field ordering ($\kpar \ll \kperp$) \citep{Strauss1976,Schekochihin2009}. Unlike phenomenological closures, this asymptotic reduction preserves the conservative structure of the parent gyrokinetic theory while dramatically simplifying the computational problem. In this regime, \Alfvenic\ and compressive fluctuations decouple. The \Alfvenic\ cascade is described by Elsasser fields representing counter-propagating \Alfven\ wave packets, evolving according to reduced MHD \citep{Howes2006}. Compressive fluctuations do not back-react on the \Alfvenic\ dynamics but retain kinetic evolution through a drift-kinetic equation describing their advection by the \Alfvenic\ turbulence and phase mixing along magnetic field lines. This decoupling enables efficient numerical treatment while maintaining research-grade accuracy for phenomena ranging from solar wind heating to tokamak microturbulence.

Existing KRMHD and gyrokinetic codes provide comprehensive, production-ready tools for turbulence research. AstroGK \citep{Numata2010} pioneered astrophysical applications with extensive validation against analytical theory and nonlinear benchmarks. Viriato \citep{Loureiro2016} employs Fourier-Hermite spectral methods for KRMHD with demonstrated accuracy in cascade and dissipation physics. These codes, along with gyrokinetic solvers like GS2 \citep{Dorland2000} and GENE \citep{Jenko2000}, represent mature platforms supporting research programs across multiple institutions. Their strength lies in comprehensive physics modules, extensive testing, and sustained development over decades. However, these capabilities require significant computational infrastructure---supercomputing allocations, specialized compilation toolchains, and domain expertise in high-performance computing. For many researchers, particularly solo investigators, small research groups, and those exploring new parameter regimes, this infrastructure barrier limits access to KRMHD turbulence research. Recent trends toward accessible simulation tools, exemplified by GX's GPU-native implementation \citep{Mandell2024} and TORAX's differentiable transport solver \citep{Citrin2024} and JAX-based kinetic simulations \citep{Joglekar2022}, demonstrate growing recognition that broadening participation requires lowering computational barriers. GANDALF extends this philosophy to KRMHD turbulence, providing spectral accuracy on commodity hardware without replacing existing production codes but rather complementing them for rapid prototyping, parameter surveys, and educational applications.

GANDALF\footnote{The name reflects the code's physics content: \textbf{G}-and-\textbf{Alf}, combining the distribution function $g$ representing slow modes (compressive fluctuations) with \Alfven\ waves.} employs JAX \citep{JAX2018} for hardware-agnostic spectral solution of the KRMHD equations. JAX provides just-in-time compilation to machine code, automatic differentiation, and transparent execution across CPU, GPU, and TPU architectures without platform-specific programming. We chose the Python/JAX ecosystem for its accessibility and growing adoption in scientific computing, eliminating CUDA dependencies while maintaining research-grade numerical accuracy. The solver implements Fourier spectral discretization in the perpendicular plane ($x$, $y$), capturing turbulent cascade dynamics with exponential convergence for smooth solutions. Velocity space employs Hermite polynomial expansion \citep{Grad1949}, providing spectral accuracy in parallel velocity $\vpar$ while allowing controllable truncation of the moment hierarchy. Time integration uses the GANDALF integrating factor method---a second-order Runge-Kutta scheme with exact treatment of linear \Alfven\ wave propagation---combined with $2/3$-rule dealiasing to control nonlinear aliasing errors. This combination runs efficiently on consumer hardware---Apple Silicon laptops, desktop GPUs, cloud TPUs---enabling workflows from rapid parameter exploration to production turbulence simulations, with potential for future integration of differentiable physics for optimization and machine learning applications.

We verify GANDALF's accuracy through a benchmark suite spanning linear, nonlinear, and turbulent regimes. Linear tests confirm correct \Alfven\ wave dispersion against analytical predictions. The Orszag-Tang vortex validates nonlinear dynamics through comparison with established MHD results. Turbulent decay simulations demonstrate convergence of energy spectra to the expected $\kperp^{-5/3}$ inertial range scaling characteristic of strong \Alfvenic\ turbulence. Velocity-space benchmarks verify the Hermite moment cascade driven by phase mixing, reproducing the expected $m^{-1/2}$ spectral scaling. These benchmarks establish that spectral methods in JAX achieve accuracy comparable to traditional implementations while executing on widely available hardware.

This paper proceeds as follows. Section~\ref{sec:formulation} presents the KRMHD equations and Hermite moment expansion employed by GANDALF. Section~\ref{sec:numerics} describes the spectral discretization, time integration scheme, and convergence properties. Section~\ref{sec:implementation} details the JAX implementation, including parallelization strategy and performance characteristics. Section~\ref{sec:verification} reports benchmark results demonstrating code accuracy across physical regimes. Section~\ref{sec:discussion} interprets the benchmark results, positions GANDALF within the existing code ecosystem, and discusses implications for accessibility. Section~\ref{sec:conclusions} summarizes our findings and outlines future development directions. By documenting GANDALF's approach, we aim to lower barriers to KRMHD turbulence research while maintaining the rigor required for quantitative plasma physics.
